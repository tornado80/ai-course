\documentclass{article}
\usepackage{setspace}
\usepackage{amsmath}
\usepackage{xcolor}
\usepackage{qtree}
\usepackage{tikz}
\usepackage{hyperref}
\usepackage{tabularx}
\usepackage{graphicx}
\usepackage{amssymb}
\usepackage{listings}
\usepackage[margin=1in]{geometry}
\usepackage{xepersian}
\settextfont{Yas}
\setmathdigitfont{Yas}
\onehalfspacing
\title{
	تمرین پنجم هوش مصنوعی
}
\author{
	امیرحسین رجبی (۹۸۱۳۰۱۳)
}
\renewcommand{\labelenumi}{\alph{enumi})}
\lstset{
	language=Python, 
	basicstyle=\ttfamily, 
	tabsize=4, 
	frame=single,
	commentstyle=\itshape\color{lightgray},
	keywordstyle=\bfseries\color{blue},
	identifierstyle=\color{black},
	stringstyle=\color{red}, 
	numbers=left
}
\bibliographystyle{plain}
\hypersetup{
	colorlinks=true,
	linkcolor=blue,
	filecolor=blue,
	citecolor = black,      
	urlcolor=cyan,
}
\newcommand{\code}[1]{\lr{\lstinline|#1|}}
\onehalfspacing
\begin{document}
	\maketitle
	
	\section*{
	سوال اول
	}
	در روش 
	\lr{model-checking}
	همه مدل‌ها بررسی می‌شوند و آنهایی که $KB$ در آنها برقرار است پاسخ مسئله است. اگر نماد‌های $P_r$، $P_b$  و $P_g$ به ترتیب به معنی وجود راه آزادی پشت در مربوطه باشند، ۸ مقدار دهی ممکن وجود دارد که از این میان تنها ۳ مقدار دهی که در آنها فقط یک نماد درست است ممکن است. (به دلیل وجود گزاره 
	$(P_r \wedge \neg P_b \wedge \neg P_g) \vee (\neg P_r \wedge  P_b \wedge \neg P_g) \vee (\neg P_r \wedge \neg P_b \wedge  P_g)$
	در پایگاه دانش)
	اکنون سه حالت را بررسی می‌کنیم. 
	\begin{enumerate}
		\item 
		$P_r \equiv True$:
		این حالت ممکن \underline{نیست} زیرا دستور هر سه در درست خواهند بود.
		\item 
		$P_g \equiv True$:
		این حالت ممکن \underline{است} چرا که دستور در سبز و آبی درست بوده و دستور در قرمز نادرست است.
		\item 
		$P_b \equiv True$
		این حالت ممکن \underline{نیست} چرا که دستور هر سه در نادرست خواهند بود.
	\end{enumerate}

	پس راه آزادی پشت در سبز است. (دقت کنید دستور درها را به منطق گزاره‌ای ترجمه نکردیم ولی مانند شرط اول انجام می‌شود.)
	\section*{
	سوال دوم
	}

ابتدا همه جملات $KB$ را به فرم نرمال عطفی
\LTRfootnote{Conjunctive Normal Form (CNF)}
تبدیل می‌کنیم. خواهیم داشت:
\begin{align*}
	V \Rightarrow W &\equiv \neg V \vee W \\
	P \Rightarrow Q &\equiv \neg P \vee Q \\
	S \Rightarrow (U \vee T) &\equiv \neg S \vee U \vee T \\
	(P \wedge R) \Rightarrow S &\equiv \neg P \vee \neg R \vee S 
\end{align*}
اکنون سعی می‌کنیم به کمک قواعد 
\lr{resolution}
نشان دهیم 
$KB \models S$
یا معادلا با فرض
$KB \wedge \neg S$
به تناقض برسیم. پس جملات زیر را داریم:
$$\boxed{P} \; \boxed{V \vee T} \; \boxed{\neg P \vee U} \; \boxed{R \vee \neg Q} \; \boxed{\neg V \vee W} \; \boxed{\neg P \vee Q} \; \boxed{\neg S \vee U \vee T} \; \boxed{\neg P \vee \neg R \vee S} \; \boxed{\neg S}$$
درخت زیر نحوه استدلال از جملات بالا را از برگ‌ها به سمت ریشه درخت نشان می‌دهد. با توجه به درخت به جمله تهی رسیده‌ایم و در نتیجه فرض خلف باطل بوده و $S$ با فرض برقراری $KB$ درست است پس $KB \models S$.
\begin{figure*}[h]
	\setLTR
	\Tree 
	[.$\boxed{}$
		[.$\boxed{\textcolor{red}{S}}$
			[.$\boxed{S \vee \textcolor{red}{\neg P}}$
				[.$\boxed{\textcolor{red}{R}}$
					[.$\boxed{\textcolor{red}{Q}}$
						{$\boxed{\textcolor{red}{\neg P} \vee Q}$}
						{$\boxed{\textcolor{red}{P}}$}
					]
					{$\boxed{\textcolor{red}{\neg Q} \vee R}$}
				]
				{$\boxed{\neg P \vee \textcolor{red}{\neg R} \vee S}$}
			]
			{$\boxed{\textcolor{red}{P}}$} 
		] 
		{$\boxed{\textcolor{red}{\neg S}}$} 
	]
\end{figure*}	
\section*{
	سوال سوم
}
ابتدا همه جملات $KB$ را به فرم نرمال عطفی تبدیل می‌کنیم. خواهیم داشت:
\begin{align}
	A \iff (B \vee E) \equiv \Big(A \Longrightarrow (B \vee E)\Big) \wedge \Big(A \Longleftarrow (B \vee E)\Big) &\equiv \stackrel{S_{1}^{1}}{(\neg A \vee B \vee E)} \wedge \stackrel{S_{1}^{2}}{(A \vee \neg B)} \wedge \stackrel{S_{1}^{3}}{(A \vee \neg E)} \tag{$S_1$} \\
	E \Longrightarrow D &\equiv \neg E \vee D \tag{$S_2$} \\
	C \wedge F \Longrightarrow \neg B &\equiv \neg C \vee \neg F \vee \neg B \tag{$S_3$} \\
	E \Longrightarrow B &\equiv \neg E \vee B \tag{$S_4$} \\
	B \Longrightarrow F &\equiv \neg B \vee F \tag{$S_5$} \\
	B \Longrightarrow C &\equiv \neg B \vee C \tag{$S_6$}
\end{align}
در هر فراخوانی تابع $DPLL$ ابتدا نماد‌های \lr{pure} در صورت وجود حذف می‌شوند. اگر چنین نماد‌هایی وجود نداشته باشند جملات واحد یا \lr{unit clause} حذف می‌شوند. در غیر این صورت برای مقدار دهی‌های مختلف یک نماد حالت بندی می‌شود و تابع دوباره صدا زده می‌شود. تابع در ابتدا به صورت زیر فراخوانی می‌شود:
$$DPLL(clauses = \{S_{1}^{1}, S_{1}^{2}, S_{1}^{3}, S_2, S_3, S_4, S_5, S_6\}, symbols = \{A, B, C, D, E, F\}, model=\{\})$$
در ادامه تنها محتویات $symbols$ و $model$ را نشان می‌دهیم. در این فراخوانی نماد‌ $D$ خالص است زیرا فقط با همین علامت در جملات دیده می‌شود و تابع به صورت 
$DPLL(clauses, \{A, B, C, E, F\}, \{D = true\})$
فراخوانی می‌شود. (دقت کنید جمله $S_2$ دیگر نادیده گرفته می‌شود.) در فراخوانی جدید نماد خالص یا جمله واحد نداریم و در نتیجه دو فراخوانی
$DPLL(clauses, \{A, C, E, F\}, \{D = true, B = true\})$
و
$DPLL(clauses, \{A, C, E, F\}, \{D = true, B = false\})$
انجام می‌شود. در فراخوانی اول، جملات $S_{1}^{1}$ و $S_4$ نادیده گرفته می‌شوند. همچنین جملات 
$S_{1}^{2}$
به $A$ ساده می‌شود؛ جمله 
$S_3$
به 
$\neg C \vee \neg F$
ساده می‌شود و جملات $S_5$ و $S_6$ به ترتیب به $F$ و $C$ ساده می‌شوند. سپس نماد $A$ خالص خواهد بود چرا که فقط به صورت $A$ و $A \vee \neg E$ ظاهر شده است. پس فراخوانی 
$$DPLL(clauses, \{C, F\}, \{D = true, B = true, A = true\})$$
انجام می‌شود. جملات به صورت $C$، $F$ و $\neg C \vee \neg F$ خواهند بود. نماد خالصی وجود ندارد اما جمله $C$ واحد است. پس فراخوانی 
$DPLL(clauses, \{F\}, \{D = true, B = true, A = true, C = true\})$
انجام شده و هنگام اجرای آن جمله 
$\neg C \vee \neg F$
به $\neg F$ ساده می‌شود. دوباره نماد خالصی وجود ندارد ولی جمله $F$ واحد است و با فراخوانی زیر 
$$DPLL(clauses, \{\}, \{D = true, B = true, A = true, C = true, F = true\})$$
جمله 
$\neg F$ 
نادرست خواهد بود و $false$ برگردانده می‌شود.

اکنون به فراخوانی
$DPLL(clauses, \{A, C, E, F\}, \{D = true, B = false\})$
برمی‌گردیم. در این صورت جملات $S_{1}^{2}$، $S_3$، $S_5$ و $S_6$ نادیده گرفته می‌شوند و جمله $S_4$ به $\neg E$ ساده شده و جمله $S_{1}^{1}$ به $\neg A \vee E$ ساده می‌شود. نماد خالصی وجود ندارد ولی جمله 
$\neg E$
واحد است. پس فراخوانی 
$$DPLL(clauses, \{A\}, \{D = true, B = false, E = false\})$$
انجام می‌شود. جملات $\neg E$ و $A \vee \neg E$ نادیده گرفته می‌شوند و جمله $\neg A \vee E$ به $\neg A$ ساده می‌شود. نماد $A$ خالص است و فراخوانی 
$$DPLL(clauses, \{\}, \{D = true, B = false, E = false, A = false\})$$
انجام می‌شود و همه جملات براساس مقدار دهی $model$ درست هستند پس $true$ برگردانده می‌شود و با انتشار به بالا اعلام می‌شود که جملات داده شده $satisfiable$ هستند.
\section*{
سوال چهارم
}
الگوریتم 
$DPLL$
پس از آنکه جملات را از نظر وجود نماد‌های خالص یا \lr{pure symbols} ساده می‌کند سراغ جملات واحد یا \lr{unit clauses} می‌رود. اگر پایگاه دانش فقط \lr{horn clause} داشته باشد یعنی جملات به فرم 
$P_1 \wedge \dots \wedge P_n \Longrightarrow Q$
یا 
$R$
که $P_i$، $Q$ و $R$ نماد هستند. در این صورت الگوریتم جملات به فرم $R$ را جملات واحد در نظر می‌گیرد و به آنها مقدار $true$ می‌دهد. سپس سراغ جملات شرطی رفته و با ساده شدن عبارات مربوط به مقدم آنها بخش تالی تنها شده و از طریق جملات واحد 
$true$
مقدار دهی می‌شوند. پس $DPLL$ روی \lr{horn clause} مانند \lr{forward chaining} عمل می‌کند. به عنوان مثال روی پایگاه دانش زیر
$$A, B, A \wedge B \Rightarrow L, A \wedge P \Rightarrow L, B \wedge L \Rightarrow M, L \wedge M \Rightarrow P, P \Rightarrow Q$$
که برگرفته از شکل 7.16 کتاب است الگوریتم به 
$A$
و  
$B$ 
مقدار $true$ نسبت می‌دهد. سپس جمله
$A \wedge B \Rightarrow L \equiv \neg A \vee \neg B \vee L$
به $L$ ساده می‌شود و در یک فراخوانی دیگر به عنوان حمله واحد شناخته می‌شود و $true$ مقدار دهی می‌شود. سپس به صورت مشابه
$B \wedge L \Rightarrow M$
به $M$ ساده شده و $M = true$ و از طریق 
$L \wedge M \Rightarrow P$
به $P =true$ و در نهایت از $P \Rightarrow Q$ به $Q = true$ می‌رسیم.
\section*{
سوال پنجم
}
\begin{enumerate}
	\item 
	با جایگذاری اصل ۱ و ۲ و اصل ۷ خواهیم داشت 
	$0 + 7 \leq 3 + 9$.
	سپس با جایگذاری ۷ در صور عمومی اصل ۴ خواهیم داشت
	$7 \leq 7 + 0$.
	با جایگذاری ۰ و ۷ در صور عمومی اصل ۶ داریم
	$7 + 0 \leq 0 + 7$.
	با جایگذاری $7 \leq 7 + 0$ و $7 + 0 \leq 0 + 7$ در اصل ۸ خواهیم داشت
	$7 \leq 0 + 7$.
	در نهایت با جایگذاری $7 \leq 0 + 7$ و $0 + 7 \leq 3 + 9$ در اصل ۸ خواهیم داشت 
	$7 \leq 3 + 9$.
\end{enumerate}
\section*{
سوال ششم
}
\begin{enumerate}
	\item 

رونیکا، درنیکا و آرنیکا را به ترتیب با نمادهای $R$، $D$ و  $A$ نشان می‌دهیم. جملات اتمی زیر را داریم: ($Member$ گزاره‌ای به معنای عضو باشگاه البرز بودن است.)
$$Member(D), Member(R), Member(A)$$
از طرفی جملات مرکب زیر را نیز داریم: ($Hiker$ و $Skier$ گزاره‌هایی به معنای کوه نورد و اسکی باز بودن است.)
$$\forall x \; \; \; Member(x) \Longrightarrow (Hiker(x) \vee Skier(x))$$
و همچنین: ($Likes$ گزاره‌ای به معنای این است که ورودی اول، ورودی دوم را دوست دارد. در اینجا $Snow$ و $Rain$ مانند $A$ و $D$ ثابت هستند.) 
\begin{align}
	\label{rules}
	&\forall y \; \; \; Skier(y) \Longrightarrow Likes(y, Snow) \\
	&\forall z \; \; \; Hiker(z) \Longrightarrow \neg Likes(z, Rain) \\
	&\forall s \; \; \; Likes(A, s) \Longrightarrow \neg Likes(D, s) \\
	&\forall t \; \; \; \neg Likes(A, t) \Longrightarrow Likes(D, t)
\end{align}
د نهایت جمله زیر را داریم:
$$\neg Likes(A, Rain) \wedge \neg Likes(A, Snow)$$

سوال گفته شده نیز به دنبال مصداق گزاره زیر است:
$$\exists x \; \; \; Member(x) \wedge Hiker(x) \wedge \neg Skier(x)$$
که از $\neg Likes(A, Rain) \wedge \neg Likes(A, Snow)$ متوجه می‌شویم 
$\neg Skier(A)$
پس خواهیم داشت 
$Hiker(A)$.
در نتیجه مصداق صور وجودی آرنیکا است.
\end{enumerate}
\end{document}