\documentclass{article}
\usepackage{setspace}
\usepackage{amsmath}
\usepackage{xcolor}
\usepackage{tikz}
\usepackage{hyperref}
\usepackage{tabularx}
\usepackage{amssymb}
\usepackage{listings}
\usepackage[margin=1in]{geometry}
\usepackage{xepersian}
\settextfont{Yas}
% Fixture for Xepersian 23 bug of setting persian math digit fonts
\ExplSyntaxOn\cs_set_eq:NN\etex_iffontchar:D\tex_iffontchar:D\ExplSyntaxOff
\setmathdigitfont{Yas}
\onehalfspacing
\title{
	تمرین دوم هوش مصنوعی
}
\author{
	امیرحسین رجبی (۹۸۱۳۰۱۳)
}
\renewcommand{\labelenumi}{\alph{enumi})}
\lstset{
	language=Python, 
	basicstyle=\ttfamily, 
	tabsize=4, 
	frame=single,
	commentstyle=\itshape\color{lightgray},
	keywordstyle=\bfseries\color{blue},
	identifierstyle=\color{black},
	stringstyle=\color{red}, 
	numbers=left
}
\bibliographystyle{plain}
\hypersetup{
	colorlinks=true,
	linkcolor=blue,
	filecolor=blue,
	citecolor = black,      
	urlcolor=cyan,
}
\newcommand{\code}[1]{\lr{\lstinline|#1|}}
\begin{document}
	\maketitle
	\section*{
		سوال اول
	}
	\begin{enumerate}
		\item 
		معادل الگوریتم 
		\lr{Random walk}
		خواهد بود زیرا اگر $x$ تنها حالت جمعیت باشد، دو حالت والد همان $x$ خواهند بود و در نتیجه پس از
		\lr{Cross over}
		فرزند همان $x$ خواهد شد و با جهش در یکی از حروف $x$ به یکی از همسایه‌های $x$ خواهیم رفت که دقیقا همان قدم زن تصادفی است.
		
	\item 
	معادل الگوریتم 
	\lr{Hill Climbing}
	خواهد بود زیرا در واقع بهترین همسایه حالت فعلی را انتخاب می‌کنیم و سراغ آن می‌رویم.
	\item 
	اگر $k$ بزرگ شود الگوریتم به این صورت خواهد بود که بسیاری از حالات و همسایه‌های آنها مورد بررسی قرار می‌گیرند و شبیه 
	\lr{Brute force}
	می‌شود که دائما تعداد زیادی از بهترین‌ها را انتخاب و وجود حالت هدف را در آن ها بررسی می‌کند.
	\item 
	اگر دما همواره صفر باشد هیچ گاه به حالتی با ارزش کمتر نخواهیم رفت و دقیقا شبیه 
	\lr{Hill Climbing}
	رفتار می‌کند. (دقت کنید لازم است شظر خاتمه الگوریتم تغییر کند به این که هیچ همسایه بهتری وجود ندارد و اگرنه با شرط فعلی که صفر بودن $T$ است الگوریتم از حلقه خارج می‌شود و جواب تصادفی نخست را برمی‌گرداند.)
	\item
	الگوریتم به سرعت به یک مینیموم یا ماکسموم محلی همگرا می‌شود چرا که الگوریتم تنها در صورتی از تپه پایین خواهد آمد که
	$\Delta E$
	کوچک باشد (هم مرتبه $T$ باشد) و برای کاهش ارتفاع زیاد، ($-\Delta E$ بزرگ) عملیات پایین آمدن 
	\lr{Reject}
	شده و فقط با مشاهده همسایه‌ها با ارزش بزرگتر، از تپه بالا می‌رود در نتیجه به سرعت مانند 
	\lr{Hill Climbing}
	رفتار خواهد کرد. اگر $T$ ثابت باشد الگوریتم از ابتدا تنها زمانی سراغ پایین آمدن از تپه می‌رود که 
	$\Delta E$
	بسیار کوچک باشد یعنی حول همان جوابی که هست یا باقی می‌ماند یا سراغ جواب های بهتر می‌رود ولی خطر کاهش ارتفاع زیاد را به جان نمی‌خرد. تقریبا شبیه 
	\lr{Hill Climbing}
	عمل خواهد کرد.
	\end{enumerate}

	\section*{
		سوال دوم
	}
	\begin{enumerate}
		\item 
	 	همه جایگشت های $n$ کلمه خواهد بود. اگر $n$ کلمه متمایز باشند می‌شود $n!$ و اگر $i$ کلمه متمایز داشته باشیم و  تکرار هر کدام 
	 	$t_1, \dots , t_i$
	 	باشند برابر
	 	$\frac{n!}{t_1!t_2!\cdots t_n!}$
	 	خواهد بود.
	 	\item 
	 	یک نوع می‌توان همسایگی بین جملات تعریف کرد به این صورت که دو جمله همسایه باشند اگه جای دو کلمه آنها عوض شده باشد. مثلا جملات «این است مجازی ترم هوش مصنوعی» و «هوش مصنوعی است این ترم مجازی» همسایه جمله داده شده خواهند بود.
	 	\item 
	 	خیر زیرا اساسا الگوریتم 
	 	\lr{Hill Climbing}
	 	الگوریتمی
	 	\lr{Complete}
	 	نیست و ممکن است در مینیموم و ماکسیموم‌های محلی و فلات‌ها گیر کند و عملا هیچ‌گاه به نقطه بهین سراسری نرسد.
	 	\item 
	 	اگر دو جمله «این است مجازی ترم هوش مصنوعی» و «هوش مصنوعی است این ترم مجازی» در مرحله
	 	\lr{Selection}
	 	به عنوان والد انتخاب شده باشند، نتیجه عملیات 
	 	\lr{Cross over}
	 	با محل شکست پس از دومین کلمه، می‌تواند فرزندان روبرو باشد: «این است این ترم مجازی» و «هوش مصنوعی است مجازی ترم هوش مصنوعی».
	 	
	\end{enumerate}

\end{document}